\documentclass[a4paper,12pt]{article}
\usepackage{amsmath}
\usepackage{blindtext}
\usepackage{listings}
\usepackage[utf8]{inputenc}
\usepackage[T1]{fontenc}
\usepackage[french]{babel}
\usepackage{fullpage}
\usepackage{color}
\usepackage[table]{xcolor}

\title{Projet d'Algorithmique II : Un problème de tomographie discrète}
\date{2023\\ Novembre}
\author{LU3IN003 : Groupe 3 \\ PINHO FERNANDES Enzo - 21107465 \\ DURBIN Deniz Ali - 21111116}
\newtheorem{exo}{Question}

\begin{document}
\maketitle
\tableofcontents
\newpage

\definecolor{darkWhite}{rgb}{0.94,0.94,0.94}
 
\lstset{
  aboveskip=3mm,
  belowskip=-2mm,
  backgroundcolor=\color{darkWhite},
  basicstyle=\footnotesize,
  breakatwhitespace=false,
  breaklines=true,
  captionpos=b,
  commentstyle=\color{red},
  deletekeywords={...},
  escapeinside={\%*}{*)},
  extendedchars=true,
  framexleftmargin=16pt,
  framextopmargin=3pt,
  framexbottommargin=6pt,
  frame=tb,
  keepspaces=true,
  keywordstyle=\color{blue},
  language=Python,
  literate=
  {²}{{\textsuperscript{2}}}1
  {⁴}{{\textsuperscript{4}}}1
  {⁶}{{\textsuperscript{6}}}1
  {⁸}{{\textsuperscript{8}}}1
  {€}{{\euro{}}}1
  {é}{{\'e}}1
  {è}{{\`{e}}}1
  {ê}{{\^{e}}}1
  {ë}{{\¨{e}}}1
  {É}{{\'{E}}}1
  {Ê}{{\^{E}}}1
  {û}{{\^{u}}}1
  {ù}{{\`{u}}}1
  {â}{{\^{a}}}1
  {à}{{\`{a}}}1
  {á}{{\'{a}}}1
  {ã}{{\~{a}}}1
  {Á}{{\'{A}}}1
  {Â}{{\^{A}}}1
  {Ã}{{\~{A}}}1
  {ç}{{\c{c}}}1
  {Ç}{{\c{C}}}1
  {õ}{{\~{o}}}1
  {ó}{{\'{o}}}1
  {ô}{{\^{o}}}1
  {Õ}{{\~{O}}}1
  {Ó}{{\'{O}}}1
  {Ô}{{\^{O}}}1
  {î}{{\^{i}}}1
  {Î}{{\^{I}}}1
  {í}{{\'{i}}}1
  {Í}{{\~{Í}}}1,
  morekeywords={*,...},
  numbers=left,
  numbersep=10pt,
  numberstyle=\tiny\color{black},
  rulecolor=\color{black},
  showspaces=false,
  showstringspaces=false,
  showtabs=false,
  stepnumber=1,
  stringstyle=\color{gray},
  tabsize=4,
  title=\lstname,
}

%%%%%%%%%%%%%%%%%%%%%%%%%%%%%%%%%%%%%%%%%%%%%%%%%%%%%%%%%%

\section{Méthode incomplète de résolution}

%%%%%%%%%%%%%%%%%%%%%%%%%%%%%%%%%%%%%%%%%%%%%%%%%%%%%%%%%%

\subsection{Première étape}

\begin{exo}
	Si l'on a calculé tous les $T(j,l)$, comment savoir s'il est possible de colorier la ligne $l_i$ entière avec la séquence entière ?
\end{exo}

Il est possible de colorier la ligne $l_i$ avec la séquence entière en vérifiant si $T(M-1,k)$ est défini comme vrai. En effet, ce dernier vérifie s'il est possible de colorier les $M$ premières cases de la ligne $l_i$, autrement dit toute la ligne, avec la séquence complète $(s_1,...,s_k)$.\\

\noindent\rule{\textwidth}{1pt}

%%%%%%%%%%

\begin{exo}
	Pour chacun des cas de base 1, 2a et 2b, indiquez si $T(j,l)$ prend la valeur vrai ou faux, éventuellement sous condition.
\end{exo}

Pour commencer, formulons les règles générales dans une formule. Afin que $\forall j \in {1,...,M-1}, \forall l \in {1,...,k}, T(j,l)$ soit défini comme vrai, il faut que deux conditions soient remplies.
\begin{itemize}
	\item Les $l$ premiers blocs de la séquence sont placés dans les j+1 premières cases de la ligne.
	\item Il doit y avoir exactement $l-1$ cases blanches, dans les j+1 premières cases, qui serviront de séparateur de blocs.\\
\end{itemize}

Avec cela, nous pouvons constater que cette formule doit être respectée dans n'importe quel cas : $j+1 \geq (\sum_{i=1}^{l} s_i) + l - 1$\\
Isolons $s_l$ de la somme, et $j$ du reste pour plus de maniabilité, et nous nous retrouvons avec la formule suivante :
$$j \geq (\sum_{i=1}^{l-1} s_i) + (s_l - 1) + (l-1)$$

\begin{description}
	\item[1. Cas $l=0$ (pas de bloc), $j \in \{0,...,M-1\}$] :\\ 
	Il n'y a pas de bloc à placer dans la séquence, par conséquent nous pouvons colorier toutes les cases en blanc.\\
	\fbox{$T(j,l)$ est donc vrai $\forall j \in \{0,...,M-1\}$ si $l = 0$.}

	\item[2. Cas $l \geq 0$ (au moins un bloc)] :
	\begin{description}
		\item[a. $j < s_l -1$] :\\
		Nous savons que $l \geq 1$ dans ce cas, par conséquent d'après la formule à respecter, nous aurions : $j \geq (\sum_{i=1}^{l-1} s_i) + (s_l - 1) + (l-1) \geq s_l - 1$\\
		Nous avons une contradiction, étant donné que nous sommes censés avoir : $j < s_l - 1$.\\
		\fbox{$\forall l \geq 1, T(j,l)$ est faux si $j < s_l - 1$.}

		\item[b. $j = s_l -1$] :\\ 
		Afin de répondre, nous devons séparer deux sous-cas distincts :
		\begin{itemize}
			\item \underline{Cas $l = 1$ :} $j \geq (\sum_{i=1}^{l-1} s_i) + (s_l - 1) + (l-1) = s_l - 1$\\
			Toutes les conditions sont respectées.\\
			\fbox{$T(j,l)$ est vrai si $j = s_l - 1$ et $l = 1$.}

			\item \underline{Cas $l > 1$ :} $j \geq (\sum_{i=1}^{l-1} s_i) + (s_l - 1) + (l-1) > s_l - 1$\\
			Il y a contradiction entre $j = s_l - 1$ et $j > s_l - 1$.\\
			\fbox{$T(j,l)$ est faux si $j = s_l - 1$ et $l > 1$.}
		\end{itemize}
	\end{description}
\end{description}

\noindent\rule{\textwidth}{1pt}

%%%%%%%%%%

\begin{exo}
	Exprimez une relation de récurrence permettant de calculer $T(j,l)$ dans le cas 2c en fonction de deux valeurs $T(j',l')$ avec j' $<$  j et l' $\leq$ l.
\end{exo}

Nous avons deux possibilités pour la case $(i,j)$, il suffit qu'une des deux soit vérifiée :
\begin{description}
	\item[1. La case $(i,j)$ est blanche] :\\
	La case est blanche, par conséquent il faut vérifier si nous pouvons placer le dernier bloc $s_l$ à la case précédente $j-1$, avec la même séquence. Il faut donc vérifier que $T(j-1, l)$ soit vrai.

	\item[2. La case $(i,j)$ est noire] :\\
	La case est noire, par conséquent, le bloc $s_l$ occupe les $s_l$ dernières cases. La case précédant le bloc $s_l$, si elle existe, sera coloriée en blanc. Il faut donc vérifier que $T(j-s_l-1, l-1)$ soit vrai.
\end{description}

La relation de récurrence est  donc : 
\fbox{$T(j,l) = T(j-1,l) \text{ OU } T(j-s_l-1, l-1)$}\\

\noindent\rule{\textwidth}{1pt}

%%%%%%%%%%

\begin{exo}
	Codez l'algorithme, puis testez-le.
\end{exo}

Afin d'optimiser le code, nous avons utilisé la programmation dynamique. Nous enregistrons chaque résultat d'appels récursifs dans une matrice.\\\\
Le code source est src/partie1/Q4\_isColorable.py\\
Le fichier test est src/partie1/Q4\_isColorableTest.py\\

\newpage

\begin{lstlisting}
def isColorable(j : int, l : int, s : list[int], memo : list[list[int]] = None) -> bool :
    """
        Renvoie vrai s'il est possible de colorier les j+1 premières cases (i,0), ..., (i,j) de la ligne l_i avec la sous-séquence (s_1, ..., s_l) des l premiers blocs de la ligne_i.
    
        Précondition : j = 0, ..., M-1 ; l = 1, ..., k; s[i] > 0 pour tout i, memo = None.
    """
    
    # Si c'est la premier appel à la fonction, on crée une matrice afin de sauvegarder en mémoire les résultats des appels récursifs.
    if memo == None :
        memo = [[None] * (l+1) for _ in range(j+1)]
        
    # On vérifie si on a déjà calculé T(j,l). Si c'est le cas, on renvoie directement sa valeur.
    if memo[j][l] != None :
        return memo[j][l]
    
    # Cas (1)
    if (l == 0) :
        memo[j][l] = True
        
    # Cas (2a)
    elif (j < s[-1] - 1) :
        memo[j][l] = False
        
    # Cas (2b)
    elif (j == s[-1] - 1) :
        if (l == 1) :
            memo[j][l] = True
        else :
            memo[j][l] = False
    
    # Cas (2c)  
    else :
        memo[j][l] = isColorable(j-1, l, s, memo) or isColorable(j-s[-1]-1, l-1, s, memo)
        
    return memo[j][l]

# PS : On retourne la valeur exceptionnellement au cas 1 pour éviter l'erreur => IndexError : list index out of range, dans le cas où l = 0.
\end{lstlisting}

\noindent\rule{\textwidth}{1pt}
\newpage

%%%%%%%%%%

\subsection{Généralisation}

%%%%%%%%%%

\begin{exo}
	Modifiez chacun des cas de l'algorithme précédent afin qu'il prenne en compte les cases déjà coloriées.
\end{exo}

%%%%%%%%%%

\end{document}