\documentclass[a4paper,12pt]{article}
\usepackage{newtxmath}
\usepackage{blindtext}

\title{Projet d'Algorithmique II : Un problème de tomographie discrète}
\date{2023\\ Novembre}
\author{LU3IN003 : Groupe 3 \\ PINHO FERNANDES Enzo - 21107465 \\ DURBIN Deniz Ali - 21111116}
\newtheorem{exo}{Question}

\begin{document}
\maketitle
\tableofcontents
\newpage

%%%%%%%%%%%%%%%%%%%%%%%%%%%%%%%%%%%%%%%%%%%%%%%%%%%%%%%%%%

\section{Méthode incomplète de résolution}

%%%%%%%%%%%%%%%%%%%%%%%%%%%%%%%%%%%%%%%%%%%%%%%%%%%%%%%%%%

\subsection{Première étape}

\begin{exo}
	Si l'on a calculé tous les $T(j,l)$, comment savoir s'il est possible de colorier la ligne $l_i$ entière avec la séquence entière ?
\end{exo}

\blindtext

%%%%%%%%%%

\begin{exo}
	Pour chacun des cas de base 1, 2a et 2b, indiquez si $T(j,l)$ prend la valeur vrai ou faux, éventuellement sous condition.
\end{exo}

\blindtext

%%%%%%%%%%

\begin{exo}
	Exprimez une relation de récurrence permettant de calculer T(j,l) dans le cas 2c en fonction de deux valeurs T(j',l') avec j' $\lt$ j et l' $\leq$ l.
\end{exo}

\blindtext

\end{document}